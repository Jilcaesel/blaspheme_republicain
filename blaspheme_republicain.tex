\documentclass[20pt]{extarticle}
\usepackage[french]{babel}
\usepackage{geometry}
\usepackage{sourcesanspro}
\usepackage{contour}
\usepackage{tikz}

\geometry{a4paper,landscape,hmargin=0cm,vmargin=0cm}
\pagestyle{empty}
\parindent=0pt

\definecolor{artref}{rgb}{0.29,0.37,0.51}
\definecolor{bleu}{rgb}{0.00,0.333,0.643}
\definecolor{rouge}{rgb}{0.937,0.255,0.208}
\newcommand\charliefont{\fontspec{Block Berthold}}

% https://www.legifrance.gouv.fr/codes/article_lc/LEGIARTI000006418556/2020-10-28

\begin{document}

\begin{tikzpicture}[x=1cm,y=-1cm]
\path (-29.7/2+0.1,-21/2+0.1) coordinate (TL);
\path ( 29.7/2+0.1,-21/2+0.1) coordinate (TR);
\path (-29.7/2+0.1, 21/2+0.1) coordinate (BL);
\path ( 29.7/2+0.1, 21/2+0.1) coordinate (BR);
\clip (BL) rectangle (TR);

\path (0,-7) node
 {\Huge\charliefont {\color{bleu}Abolissons} le {\color{rouge}délit} de};

\path (0,-1) node (code) {
\fboxsep=0.1em \fboxrule=0.05em %
\fbox{%
\fboxsep=1em \fboxrule=0.1em %
\fbox{%
\begin{minipage}{15cm}
\sourcesanspro
\parindent=1em
\parskip=0.5ex

\noindent\textbf{\color{artref}Code pénal, article 433-5-1}

Le fait, au cours d'une manifestation organisée ou réglementée par les
autorités publiques, d'outrager publiquement l'hymne national ou le drapeau
tricolore est puni de 7\,500 euros d'amende.

Lorsqu'il est commis en réunion, cet outrage est puni de six mois
d'emprisonnement et de 7\,500 euros d'amende.
\end{minipage}%
}%
}%
};

\newcommand\blasphemerepublicain[2]{%
\begin{scope}
\clip #1;
\path (0,5.8) node[anchor=base]
 {\Huge\charliefont#2{BLASPHÈME RÉPUBLICAIN!}};
\end{scope}
}

\blasphemerepublicain{}{\color{white}\contour{black}}
\blasphemerepublicain{(-9,8) rectangle (-3,0)}{\color{bleu}}
\blasphemerepublicain{(3.07,8) rectangle (9,0)}{\color{rouge}}

\newcommand\cfl{\charliefont}

\path (TL) ++(+3,+1.5) node[below,rotate=30,align=center] {%
 \cfl Parce que\\
 \cfl c'est vraiment pour\\
 \cfl la liberté d'expression%
};

\path (TR) ++(-3,+1.5) node[below,rotate=-30,align=center] {%
 \cfl Parce que\\
 \cfl les idées et les symboles\\
 \cfl n'ont pas besoin de protection%
};

\path (BL) ++(+3,-1.5) node[above,rotate=-30,align=center] {%
 \cfl Parce que la République\\
 \cfl accepte qu'on se moque\\
 \cfl si on respecte\\
 \cfl ses lois%
};

\path (BR) ++(-3,-1.5) node[above,rotate=30,align=center] {%
 \cfl Parce que ce n'est pas\\
 \cfl deux poids deux\\
 \cfl mesures%
};

\path (code.north west) ++(-0.3,-0.3) coordinate (CTL);
\path (code.north east) ++(+0.3,-0.3) coordinate (CTR);
\path (code.south west) ++(-0.3,+0.3) coordinate (CBL);
\path (code.south east) ++(+0.3,+0.3) coordinate (CBR);
\draw[line width=0.2em,line cap=round,bleu!50!rouge]
 (CTL) -- (CBR) (CBL) -- (CTR);

\end{tikzpicture}

\end{document}
